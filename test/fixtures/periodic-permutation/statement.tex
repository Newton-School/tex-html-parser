\textbf{Periodic Permutation}

Rahul was gifted a $permutation$ $P$ for his birthday. A $permutation$ of size $N$ is an array consisting of $all$ integers from $1$ to $N$, each occurring exactly once.

Rahul can perform the following operation as many times as he wants:

\begin{itemize}
\item Simultaneously for all $i$ $(1 \leq i \leq N)$, set $P[i] = P[P[i]]$.
\end{itemize}

Rahul wants to know the minimum number of operations required for the $permutation$ to return to its original state, or whether it is even possible to do so.

\textbf{Input}

The first line contains a single integer $T$ $(1 <= T <= 100000)$ --- the number of test cases.

Each test case consists of two lines:
\begin{itemize}
\item The first line contains a single integer $N$ $(1 <= N <= 1000000)$ --- the length of the $permutation$.
\item The second line contains $N$ space-separated integers $P_1, P_2, \dots, P_N$, $(1 <= P_i <= N)$ representing the $permutation$ $P$.
\end{itemize}

It is guaranteed that the sum of $N$ over all test cases does not exceed $10^6$.

\textbf{Output}

For each test case, print on a new line:
"--- Print $-1$ if the $permutation$ cannot return to its original state after any number of operations, else print the minimum number of operations required for the $permutation$ to return to its original state. Since this number may be large, print it modulo $100000007$.

\textbf{Example}

\begin{lstlisting}
Input
3
1
1
2
1 2
2
2 1

Output
1
1
-1
\end{lstlisting}
